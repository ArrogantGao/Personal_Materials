\documentclass{beamer}
\usepackage[utf8]{inputenc}
\usepackage{bookmark}
\usepackage{svg}
\title{Lightning Talk}
\date{November 17, 2025}
\author{\textbf{Xuanzhao Gao}}
% \institute{\inst{1} Center for Your Center, Flatiron Institute, Simons Foundation \and \inst{2} Other Author's Institute}

\usetheme{CCM}

\begin{document}

\begin{frame}
	\titlepage
\end{frame}

\begin{frame}
	\frametitle{Hi!}
	I'am Xuanzhao Gao, a new FRF working with Alex and Jason. 
	
	\vspace{0.3cm}

	I'm from Changchun, a snowy city in China.

	\begin{figure}[h]
		\begin{columns}[T]
			\begin{column}{0.3\textwidth}
				\centering
				\includegraphics[width=\linewidth]{figs/changchun.jpg}
			\end{column}
			\begin{column}{0.4\textwidth}
				\centering
				\includegraphics[width=\linewidth]{figs/changchun_2.jpg}
			\end{column}
		\end{columns}
	\end{figure}

	My education background is a little bit complex :-)
	\begin{itemize}
		\item Phd in Hong Kong University of Science and Technology, applied mathematics
		\item BSc in University of Science and Technology of China, physics and computer science
	\end{itemize}
\end{frame}

\begin{frame}
	\frametitle{Boundary Integral Equations}

	I am working with Alex and Jason on a project about solving Poisson's equation under dielectric mismatches and corner singularities.

	\begin{figure}[h]
		\centering
		\includegraphics[width=0.25\textwidth]{figs/mbox_toy.png}
	\end{figure}

	We are doing a great job on 2D cases by adaptive refinement of the corners:

	\begin{figure}[h]
		\centering
		\includegraphics[width=0.7\textwidth]{figs/mbox_fmm2d_2.png}
	\end{figure}
	
\end{frame}

\begin{frame}
	\frametitle{Fast Summation Methods}

	I am working on fast summation methods for the $1/r$ kernel.

	Now I am collaborating with Shidong and Jiuyang for a DMK based solver to achieve $O(\log{N})$ update complexity in Monte Carlo simulations of charged particles.

	\begin{figure}[h]
		\centering
		\includegraphics[width=0.5\textwidth]{figs/pdmk4mc_runtime.png}
	\end{figure}

\end{frame}

\begin{frame}
	\frametitle{Tensor Network Algorithms}

	Another direction I am interested in the tensor network algorithms, which is kind of mixture of math, quantum physics and computer science.

	Now I use it to solve combinatorial optimization problems.

	\begin{figure}[h]
		\centering
		\includegraphics[width=0.9\textwidth]{figs/bbtn_time_complexity.pdf}
	\end{figure}

	I am also trying to develop methods for quantum many-body simulations.
\end{frame}

\begin{frame}[fragile]
	\frametitle{I Code in Julia and Cpp}

	\begin{figure}[h]
		\centering
		\includegraphics[width=\textwidth]{figs/github.png}
	\end{figure}
\end{frame}

\end{document}
